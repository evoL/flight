\documentclass{mwrep}
\usepackage{polski}
\usepackage[utf8]{inputenc}
\usepackage{graphicx}
\usepackage{tabularx}
\usepackage{changepage}

\newcommand*{\titleGP}{\begingroup
\centering

{\Large Studencka Pracownia Inżynierii Oprogramowania}\par
\vspace*{16\baselineskip}

{\LARGE Rafał Hirsz, Karol Wieczorek, Krzysztof Wróbel\par}
\vspace*{\baselineskip}

\rule{\textwidth}{1.6pt}\vspace*{-\baselineskip}\vspace*{2pt}
\rule{\textwidth}{0.4pt}\\[\baselineskip]

{\Huge SYMULATOR LOTU}\\[0.2\baselineskip]

\rule{\textwidth}{0.4pt}\vspace*{-\baselineskip}\vspace{3.2pt}
\rule{\textwidth}{1.6pt}\\[\baselineskip]

\scshape
{\huge Wstępny harmonogram projektu}\par
\vspace*{2\baselineskip}

\begin{figure}[h]
\centering
\includegraphics[width=5\baselineskip]{flightsim-team-logo.pdf}
\end{figure}
\vfill

{\large Wrocław 2013}\par

\pagebreak

\endgroup}

\begin{document}

\thispagestyle{empty}
\titleGP

\begin{center}
\begin{table}[h]
\begin{center}
\begin{tabularx}{1\textwidth}{|l|l|l|X|}
\hline
Data & Wersja & Autor & Opis zmian \\ \hline
3.11.2013 & 1.0 & Zespół & Utworzenie dokumentu
\end{tabularx}
\end{center}
\vspace{3ex}
\caption{Historia zmian dokumentu}\label{T:Zmiany}
\end{table}
\end{center}

\pagebreak

\tableofcontents

\newpage
\section{Opis}
Harmonogram przedsięwzięcia został sporządzony przy użyciu powszechnie stosowanych metod szacowania rozmiaru i pracochłonności oprogramowania oraz w oparciu o wiedzę członków zespołu. Do wyznaczenia wielkości projektu użyto metody punktów funkcyjnych i na jej podstawie stwierdzono, że kod programu będzie miał ok. 50-70 tysięcy linii. Następnie przy pomocy metody COCOMO II obliczono przybliżoną pracochłonność całego projektu (wyniosła ok. 180 osobomiesięcy) oraz jego poszczególnych części składowych. Do tego wyniku dodano 20 osobomiesięcy buforu związanego z urlopami pracowników i innymi niemożliwymi do przewidzenia zdarzeniami. Przy założeniu, że zespół programistów będzie liczył 10 osób, wykonanie oprogramowania powinno zająć niecałe dwa lata.
Poniższy harmonogram przedstawia planowany przebieg prac nad projektem i ze względu na wczesną fazę prac jest obłożony dużą niedokładnością, jego dokładniejsza wersja powstanie w fazie analizy. 


\begin{figure}[!h]
	\centerline{\includegraphics*[scale=0.8]{harmonogram-tabela.pdf}}
	\caption{Harmonogram projektu}
\end{figure}

\end{document}
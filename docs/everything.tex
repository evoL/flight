\documentclass{mwrep}
\usepackage{polski}
\usepackage[polish]{babel}
\usepackage[utf8]{inputenc}
\usepackage{tgbonum}
\usepackage{graphicx}
\usepackage{tabularx}
\usepackage{hyperref}

\newcommand*{\titleGP}{\begingroup
\centering

{\large Studencka Pracownia Inżynierii Oprogramowania}\par
\vspace*{16\baselineskip}

{\Large Rafał Hirsz, Karol Wieczorek, Krzysztof Wróbel\par}
\vspace*{\baselineskip}

\rule{\textwidth}{1.6pt}\vspace*{-\baselineskip}\vspace*{2pt}
\rule{\textwidth}{0.4pt}\\[\baselineskip]

{\Huge SYMULATOR LOTU}\\[0.2\baselineskip]

\rule{\textwidth}{0.4pt}\vspace*{-\baselineskip}\vspace{3.2pt}
\rule{\textwidth}{1.6pt}\\[\baselineskip]

\scshape
{\huge Założenia ogólne}\par
\vspace*{2\baselineskip}

\begin{figure}[h]
\centering
\includegraphics[width=5\baselineskip]{flightsim-team-logo.pdf}
\end{figure}
\vfill

{\large Wrocław 2013}\par

\pagebreak

\endgroup}

\begin{document}

\thispagestyle{empty}
\titleGP

\begin{center}
\begin{table}[h]
\begin{center}
\begin{tabularx}{1\textwidth}{|l|l|l|X|}
\hline
Data & Wersja & Autor & Opis zmian \\ \hline
05.11.2013 & 1.0 & Zespół & Utworzenie dokumentu \\
\hline
\end{tabularx}
\end{center}
\vspace{3ex}
\caption{Historia zmian dokumentu}\label{T:Zmiany}
\end{table}
\end{center}

\pagebreak

\tableofcontents

\chapter{Słownik pojęć}

\begin{description}
\item[Kąt natarcia] - (ang. \emph{angle of attack}) jest to (umowny) kąt pomiędzy kierunkiem strugi napływającego powietrza, a~cięciwą skrzydła.
\item[Krawędź natarcia] - (ang. \emph{leading edge}) przednia krawędź skrzydła.
\item[Krawędź spływu] - (ang. \emph{trailing edge}) tylna część skrzydła, gdzie zbiegają się strugi powietrza znad i~spod skrzydła.
\item[Klapy] - (ang. \emph{flaps}) ruchome powierzchnie sterowe, umiejscowione zazwyczaj w~tylnej części skrzydła statku powietrznego, pozwalająca w~razie potrzeby znacznie zwiększyć siłę nośną oraz opór skrzydła.
\item[Skrzela] - (ang. \emph{slots}) elementy skrzydła samolotu umieszczone wzdłuż krawędzi natarcia, mają za zadanie poprawienie własności aerodynamicznych statku powietrznego przy locie z~małymi prędkościami i~na dużych kątach natarcia poprzez zwiększenie siły nośnej lub opóźnienie oderwania się strug.
\item[Lotki] - (ang. \emph{ailerons}) powierzchnie sterowe statku powietrznego, służące do kontroli jego przechylenia.
\item[Trymer] - (ang. \emph{trim tab}) wychylna część steru na jego krawędzi spływu, ustawiana w~locie pod odpowiednim kątem, wychylana w~przeciwnym kierunku niż ster, zmniejsza siłę potrzebną pilotowi do utrzymania steru w~stałym wychyleniu.
\item[Spojlery] - (ang. \emph{spoilers}) element skrzydła, którego wychylenie skutkuje w~zmniejszeniu siły nośnej i~zwiękrzesniu tarcia skrzydła.
\item[System odniesienia WGS-84] - zbiór parametrów (z 1984) określających wielkość i~kształt Ziemi oraz właściwości jej potencjału grawitacyjnego. System ten definiuje elipsoidę, która jest generalizacją kształtu geoidy, wykorzystywaną do tworzenia map.
\item[Wysokościomierz barometryczny] - czuły wyskalowany przyrząd wskazujący wysokość dzieki pomiarowi ciśnienia powietrza.
\item[Radiowysokościomierz] - rodzaj lotniczego przyrządu pokładowego, który wskazuje rzeczywistą wysokość lotu nad powierzchnią Ziemi.
\item[Wariometr] - przyrząd służący do wskazywania prędkości pionowej (wznoszenia lub opadania) statku.
\item[Sztuczny horyzont] - przyrząd pomagający w~określeniu orientacji przestrzennej samolotu wzgledem horyzontu.
\item[ILS] - (ang. \emph{Instrumental Landing System}) radiowy system nawigacyjny wspomagający lądowanie samolotu w~warunkach ograniczonej widoczności.
\item[VOR] - (ang. \emph{VHF Omni-directional Range}) rodzaj radiolatarni stosowanej w~lotnictwie.
\item[DME] - (ang. \emph{Distance Measuring Equipment}) radiowa pomoc nawigacyjna służąca do pomiaru fizycznej odległości pomiędzy samolotem a~stacją naziemną.
\item[NDB] - (ang. \emph{Non-Directional Beacon}) naziemna radiolatarnia bezkierunkowa.
\item[TCAS] - (ang. \emph{Traffic Alert and Collision Avoidance System}) pokładowy system zapobiegający zderzeniom statków powietrznych.
\item[TAWS] - (ang. \emph{Terrain Awareness and Warning System}) urządzenie pokładowe ostrzegające pilotów o~zbliżającej się powierzchni gruntu.
\item[Metoda punktów funkcyjnych] - najczęściej stosowana metoda szacowania wielkości systemów informatycznych uzależniająca ich złożoność od ilości realizowanych przez nie funkcji.
\item[COCOMO II] - jedna z~metod szacowania pracochłonności systemu informatycznego bazująca na rozmiarze oprogramowania.
\end{description}

\chapter{Specyfikacja produktu}

\section{Ogólna specyfikacja produktu}

Symulator lotu powinien być grą symulującą ruch cywilnych maszyn latających, przeznaczoną dla komputerów osobistych klasy PC. Jego założeniem jest jak najwierniejsze modelowanie zjawisk fizycznych towarzyszących statkowi powietrznemu w~czasie lotu, jak i~na lotnisku. Dodatkowo kładzie się nacisk na dokładne odzwierciedlenie specyfiki ruchu lotnicznego i~pracy pilota.

\section{Komponenty symulatora}

\subsection{Silnik fizyczny}

Podstawowym celem symulatora jest dokładne wymodelowanie zjawisk fizycznych. Konieczna do tego jest implementacja w~grze silnika dynamiki newtonowskiej. Po aplikacji sił działających na bryłę sztywną powinno być w~nim możliwe kontrolowanie jego toru ruchu, prędkości i~przyspieszenia. Nie zamierza się uwzględniać fizyki relatywistycznej.

\subsection{Modelowanie ruchu samolotu}

Statek powietrzny jest w~sensie fizycznym bryłą sztywną o~zmieniającym się w~czasie ruchu kształcie i~masie. Działają siły różnego pochodzenia. Można wsród nich wymienić:
\begin{description}
\item[Grawitacja ziemska] w~zależności od aktualnego położenia obiektu na Ziemi i~wysokości nad poziomem morza przyspieszenie ziemskie może mieć minimalnie różne wartości. Spowodowane to jest odległością od środka planety i~siły odśrodkowej wywołanej jej ruchem obrotowym.
\item[Siła nośna] Skrzydła samolotu poruszające się w~ściśliwym płynie takim, jakim jest powietrze, generują siłę przeciwną do siły grawitacji. Pozwala to na unoszenie się obiektu nad Ziemią. Wartość siły nośnej zależy od profilu skrzydła, jego powierzchni, prędkości w~ośrodku, kąta natarcia, ciśnienia atmosferycznego, temperatury, wychylenia klap i~skrzeli (slotów). Obroty samolotu wokół jego osi są powodowane przez różnice w~sile nośnej generowanej przez poszczególne powierzchnie. Wychylanie lotek generuje różnice siły nośnej na skrzydłach, co jest przyczyną przechylania się maszyny na boki. Usterzenie na statecznikach pionowych i~poziomych obraca obiekt odpowiednio w~osi pionowej i~wdłuż skrzydeł. Pozycja poszczególnych powierzchni sterujących może być zachowana poprzez użycie trymeru.
\item[Opór powietrza] Ruch obiektu w~atmosferze wymaga użycia siły do przesuwania napotkanych molekułów. Siła ta rośnie wraz ze wzrostem gęstości ośrodka i~prędkości przesuwania się w~nim.
\item[Ciąg] Siła ciągu jest wynikiem działania silników. Planuje się implementacje trzech różnych rodzajów napędu: śmigłowy, odrzutowy i~rakietowy. Moc ciągu zależy od tempa spalania paliwa, gęstości powietrza i~od indywidualnych właściwości silnika. Przy locie wznoszącym lub opadającym ciąg ma wpływ na siłę nośną statku.
\item[Hamowanie] Statek powietrzny może być wyhamowywany na cztery różne sposoby: poprzez użycie spojlerów, odwracaczy ciągu, hamulców w~kołach (tylko po lądowaniu) oraz poprzez odpowiednie manewrowanie całym samolotem.
\item[Kontakt z~podłożem] Podwozie samolotu działa podobnie do sprężyny. Przyziemienie powoduje aplikacje siły skierowanej pionowo w~górę.
\end{description}

\vspace*{\baselineskip}
Ważną informacją potrzebną do symulacji ruchu samolotu jest jego rozkład masy. W~szczególności na zmianę zachowania statku w~powietrzu ma wpływ ilość oraz rozmieszczenie paliwa, pasażerów i~bagaży. Podczas lotu zmienia się także ilość paliwa w~zbiornikach.

\subsection{Warunki pogodowe}
Oferowane będą trzy tryby pogody.

\begin{itemize}
\item Warunki predefiniowane przez użytkownika.
\item Warunki zgodne z~lokalnym klimatem, porą roku i~dnia.
\item Warunki aktualne (pobierane z~zewnętrznych zasobów).
\end{itemize}

\noindent w~każdym z~przypadków, wpływ na rzeczywiste odczyty temperatury, wilgotności i~ciśnienia ma wysokość, na której znajduje się obiekt.


\subsection{Wykrywanie kolizji}
Ze względów etycznych, po rozbiciu samolotu zostanie wygenerowany jedynie komunikat mówiący o~jego zniszczeniu. Nieprzewidywane są żadne animacje katastrofy i~wraku. Kolizje z~terenem poza miejscem lądowania będą wykrywane mniej dokładnie oraz będą od razu przerywały symulację lotu. W~przypadku lotnisk informacja o~kolizji będzie wyświetlana w~wyniku przeciążenia podwozia lub dotknięcia ziemi innym elementem maszyny. Zarówno zbliżenia do terenu lotniska, jak i~do innych obiektów będą symulowane z~najwyższą szczegółowością.

\subsection{Dane techniczne samolotów}
Program ma umożliwiać korzystanie z~różnych modeli samolotów. Niekoniecznie muszą one być dostarczone przez wytwórce symulatora. Dane konfiguracyjne powinny spełniać specyfikację ze strony \linebreak \url{http://msdn.microsoft.com/en-us/library/cc526949.aspx} .

\subsection{Topografia Ziemi}
Przewiduje się symulację lotu w~realnym świecie. Ukształtowanie terenu, jeziora, morza i~położenie lotnisk będą wczytywane z~map. Kształt planety powinien być elipsoidą obrotową spłaszczoną na biegunach wg systemu odniesienia WGS-84.

\subsection{Pomoce nawigacyjne}
Symulator powinien modelować zachowanie się niektórych pomocy nawigacyjnych, m.in.: mapy, kompas, wysokościomierz barometryczny, radiowysokościomierz, prędkościomierz, wariometr, sztuczny horyzont oraz systemy ILS, VOR, DME, NDB, TCAS, TAWS i~GPS.

\subsection{Generowanie grafiki}
W programie umożliwiony będzie podgląd świata z~kabiny pilota, jak i~z zewnątrz samolotu. Sceneria będzie przedstawiała podstawową rzeźbę terenu, urządzenia na lotnisku oraz inne pojazdy i~statki powietrzne. Dodatkowo podczas widoku z~kokpitu przedstawione bedą wszystkie przyrządy pokładowe.

\subsection{Interakcja z~użytkownikiem}
Użytkownik będzie mógł sterować statkiem używając myszki, joysticka i~klawiatury. Myszka będzie pomagała w~rozglądaniu się, joystick w~manualym sterowaniu samolotem, a~klawiatura w~szczegółowych ustawieniach lotu.

\subsection{Kontrola lotów}
Osobną pomocą nawigacyjną jest automatyczna kontrola lotów. Będzie ona przeprowadzana w~trybie tekstowym w~angielskim języku lotniczym.

\subsection{Inne pojazdy}
Implementacji wymaga ruch służb lotniskowych, np. cystern, samochodów, holowników, itd.

\subsection{Planowanie lotu}
Ważnym elementem przed startem maszyny jest zaplanowanie lotu. Aplikacja będzie umożliwiała graficzne zaplanowanie trasy przelotu i~ilości paliwa.

\subsection{Interfejs użytkownika}
Interfejs użytkownika powinien pozwalać w~trybie okienkowym na skonfigurowanie parametrów symulacji. w~szczególności jest to: wybór samolotu, dnia, godziny, miejsca startu, pogody, trasy lotu, ilości paliwa i~liczby pasażerów.

\subsection{Gra online}
Użytkownicy będą mogli współpracować ze sobą i~widzieć się nawzajem, grając poprzez sieć internetową.

\section{Scenariusze użycia}
Przewiduje się pięć różnych trybów gry.
\subsection{Wolny lot}
W tym trybie jedynym celem użytkownika jest pilotowanie samolotu, wykonywanie akrobacji, testowanie osiągów, trening pilotażu w~trudnych warunkach pogodowych, itp. Nie buduje się planu lotu, ani nie działa kontrola lotów. w~przestrzeni nie ma także innych statków powietrznych.
\subsection{Przelot wg planu lotu}
Scenariusz ten najbardziej odzwierciedla realną pracę pilota. Gra zaczyna się przy hangarze na płycie lotniska. Użytkownik powinien utworzyć plan lotu, zatankować, zabrać ładunek, bądź pasażerów, uzyskać pozwolenie na lot i~start, stosować się do poleceń kontroli lotów i~prawa lotniczego. Może napotkać inne, kontrolowane przez komputer, samoloty.
\subsection{Tryb wielu graczy}
Symulacja podobna jest do poprzedniej z~tą różnicą, że uczestniczyć w~niej może kilku rzeczywistych użytkowników.
\subsection{Samouczek}
Tryb treningowy pozwalający na praktykę sztuki pilotażu i~poznawanie pomocy nawigacyjnych.
\subsection{Misje}
Specjalne, z góry określone zadania do wykonania przez gracza. Za ich ukończenie przyznawane będą trofea.

\section{Opis użytkownika}
Z definicji gra jest dokładnym symulatorem i~jej celem nie jest zabawiać, tylko dokładnie modelować rzeczywistość. Od użytkownika oczekuje się precyzji, cierpliwości, opanowania, szybkiej reakcji, zdolności manualnych, jak i~chęci nauki oraz znajomości języka angielskiego.

\chapter{Harmonogram}

\section{Opis}
Harmonogram przedsięwzięcia został sporządzony przy użyciu powszechnie stosowanych metod szacowania rozmiaru i~pracochłonności oprogramowania oraz w~oparciu o~wiedzę członków zespołu. Do wyznaczenia wielkości projektu użyto metody punktów funkcyjnych i~na jej podstawie stwierdzono, że kod programu będzie miał ok. 50-70 tysięcy linii. Następnie przy pomocy metody COCOMO~II obliczono przybliżoną pracochłonność całego projektu (wyniosła ok. 180 osobomiesięcy) oraz jego poszczególnych części składowych. Do tego wyniku dodano 20 osobomiesięcy buforu związanego z urlopami pracowników i~innymi niemożliwymi do przewidzenia zdarzeniami. Przy założeniu, że zespół programistów będzie liczył 10 osób, wykonanie oprogramowania powinno zająć niecałe dwa lata.
Poniższy harmonogram przedstawia planowany przebieg prac nad projektem i~ze względu na wczesną fazę prac jest obłożony dużą niedokładnością, jego dokładniejsza wersja powstanie w~fazie analizy.


\begin{figure}[!h]
	\centerline{\includegraphics*[scale=0.8]{harmonogram-tabela.pdf}}
	\caption{Harmonogram projektu}
\end{figure}

\chapter{Kosztorys}

\section{Zespół}

Planujemy oddać część pracy w~ręce zewnętrznych podwykonawców. Dlatego dokonujemy podziału zespołu na zespół stacjonarny i~niestacjonarny.

\subsection{Zespół stacjonarny}

\begin{itemize}
\item 10 programistów C++
\item 2 kierowników zespołów
\item 4 testerów oprogramowania
\end{itemize}

\subsection{Zespół niestacjonarny}

\begin{itemize}
\item 2 grafików 2D
\item 2 grafików 3D
\end{itemize}

\section{Miesięczne wynagrodzenia brutto pracowników}

Poniższa lista została sporządzona na bazie danych pochodzących z raportu \emph{Wynagrodzenia na stanowiskach IT w~2012 roku} wydanego przez kancelarię Sedlak \& Sedlak.\footnote{\url{http://wynagrodzenia.pl}} \\

\begin{tabular}{lr}
programista C++ & 6500 zł \\
kierownik zespołu & 8000 zł \\
grafik 3D & 7400 zł \\
grafik 2D & 3700 zł \\
tester oprogramowania & 4200 zł
\end{tabular}

\section{Koszt utrzymania pracowników}

\begin{tabular}{llr}
programiści C++ & 20 miesięcy & 1 300 000 zł \\
kierownik zespołu & 20 miesięcy & 160 000 zł \\
graficy 3D & 13 miesięcy & 192 400 zł \\
graficy 2D & 13 miesięcy & 96 200 zł \\
testerzy & 20 miesięcy & 336 000 zł \\
\hline
\textbf{suma} && \textbf{2 084 600 zł}
\end{tabular}

\section{Koszt całkowity}

\begin{tabular}{lr}
wynagrodzenia & 2 084 600 zł \\
bufor na nieprzewidziane wydatki & 40 000 zł \\
\hline
\textbf{suma} & \textbf{2 124 600 zł}
\end{tabular}

\end{document}
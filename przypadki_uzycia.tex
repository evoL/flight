\documentclass{mwrep}
\usepackage{polski}
\usepackage[polish]{babel}
\usepackage[utf8]{inputenc}
\usepackage{tgbonum}
\usepackage{graphicx}
\usepackage{tabularx}
\usepackage{hyperref}

\renewcommand{\labelitemi}{$\bullet$}
\setlength{\parindent}{0cm}
\addto\captionspolish{\renewcommand{\tablename}{Tabela}}

\newcommand*{\titleGP}{\begingroup
\centering

{\large Studencka Pracownia Inżynierii Oprogramowania}\\Instytut Informatyki Uniwersytetu Wrocławskiego\par
\vspace*{16\baselineskip}

{\Large Rafał Hirsz, Karol Wieczorek, Krzysztof Wróbel\par}
\vspace*{\baselineskip}

\rule{\textwidth}{1.6pt}\vspace*{-\baselineskip}\vspace*{2pt}
\rule{\textwidth}{0.4pt}\\[\baselineskip]

{\Huge SYMULATOR LOTU}\\[0.2\baselineskip]

\rule{\textwidth}{0.4pt}\vspace*{-\baselineskip}\vspace{3.2pt}
\rule{\textwidth}{1.6pt}\\[\baselineskip]

\scshape
{\huge Analiza przypadków użycia}\par
\vspace*{2\baselineskip}

\begin{figure}[h]
\centering
\includegraphics[width=5\baselineskip]{flightsim-team-logo.pdf}
\end{figure}
\vfill

{\large Wrocław 2014}\par

\pagebreak

\endgroup}

\begin{document}

\thispagestyle{empty}
\titleGP

\begin{center}
\begin{table}[h]
\begin{center}
\caption{Historia zmian dokumentu}\label{T:Zmiany}
\vspace{3ex}
\begin{tabularx}{1\textwidth}{|l|l|l|X|}
\hline
Data & Wersja & Autor & Opis zmian \\ \hline
2014-01-10 & 1.0 & Karol Wieczorek & Utworzenie dokumentu \\
\hline
\end{tabularx}
\end{center}
\end{table}
\end{center}

\pagebreak

\tableofcontents

\chapter{Ustawienia}
\section{Wybór ustawień graficznych}
\begin{enumerate}
  \item Użytkownik klika w przycisk ,,Ustawienia'' w~ menu głównym.
  \item Użytkownik wybiera zakładkę ,,Ustawienia graficzne''.
  \item Użytkownik wprowadza jedną lub~ więcej zmian:
    \begin{itemize}
      \item rozdzielczość okna;
      \item poziom szczegółów kokpitu samolotu;
      \item poziom szczegółów map;
      \item włączenie synchronizacji pionowej;
      \item ustawienia dotyczące graficznego interfejsu użytkownika.
    \end{itemize}
  \item Użytkownik naciska przycisk ,,Wprowadź zmiany''.
  \item Program wprowadza podane przez użytkownika zmiany.
\end{enumerate}

\section{Wybór ustawień dźwieku}
\begin{enumerate}
  \item Użytkownik klika w przycisk ,,Ustawienia'' w~ menu głównym.
  \item Użytkownik zakładkę ,,Ustawienia dźwięku''.
  \item Użytkownik wprowadza jedną lub~ więcej zmian:
  \begin{itemize}
    \item głośność dźwięku;
    \item ilość kanałów.
  \end{itemize}
  \item Użytkownik naciska przycisk ,,Wprowadź zmiany''.
  \item Program wprowadza podane przez użytkownika zmiany.
\end{enumerate}

\chapter{Początek rozgrywki}
\section{Rozpoczęcie samouczka}
\begin{enumerate}
  \item Użytkownik naciska przycisk ,,Samouczek'' w~ menu głównym.
  \item Program wyświetla listę wszystkich dostępnych scenariuszy. Scenariusze, które zostały już ukończone są oznaczone zielonym kolorem, pierwsza niewykonana lekcja białym, a~ pozostałe czerwonym.
  \item Użytkownik wskazuje scenariusz, którą chciałby odbyć.
  \item Program wczytuje dane wybranego scenariusza (mapę, samolot, ustawienia, sytuację początkową).
  \item Rozpoczyna się właściwa rozgrywka.
\end{enumerate}

\section{Rozpoczęcie misji}
\begin{enumerate}
  \item Użytkownik naciska przycisk ,,Misje'' w~ menu głównym.
  \item Program wyświetla listę wszystkich dostępnych misji wraz z~ opisem i~ szacowanym czasem ukoczenia.
  \item Użytkownik wskazuje zadanie, które chciałby wykobnać.
  \item Program wyświetla ekran, na którym znajdują się wszystkie szczegóły, mapy i~ diagramy potrzebne do~ jej przejścia.
  \item Użytkownik zapoznaje się z~ ww.~ pomocami i~ naciska przycisk ,,Rozpocznij misję''.
  \item Program wczytuje dane wybranej misji (mapę, samolot, ustawienia, sytuację poczatkową).
  \item Rozpoczyna się właściwa rozgrywka.
\end{enumerate}

\section{Rozpoczęcie swobodnego lotu}
\begin{enumerate}
  \item Użytkownik naciska przycisk ,,Tryb swobodnego lotu'' w~ menu głównym.
  \item Program wyświetla ekran z~ dostępnymi samolotami.
  \item Użytkownik wybiera samolot i~ naciska przycisk ,,Dalej''.
  \item Program wyświetla ekran z~ dostępnymi mapami oraz miejscami startu znajdującymi się na~ poszczególnych mapach.
  \item Użytkownik wybiera mapę oraz~ miejsce startu i~ naciska przycisk ,,Dalej''.
  \item Program wyświetla ekran z pozostałymi ustawieniami dotyczącymi:
  \begin{itemize}
    \item warunków pogodowych
    \item ładunku samolotu
    \item planu lotu
    \item ewentualnych usterek samolotu
  \end{itemize}
  \item Użytkownik dostosowuje rozgrywkę do własnych upodobań i~ naciska przycisk ,,Dalej''.
  \item Program wyświetla podsumowanie konfuguracji lotu swobodnego.
  \item Użytkownik zapoznaje się z~ podsumowaniem. Jesli wszystko jest w~ porządku, naciska przycisk ,,Rozpocznij grę'', w~ przeciwnym wypadku naciska przycisk ,,Wstecz'' i~ wprowadza poprawki.
  \item Program wczytuje odpowiednie dane (mapę, samolot, ustawienia, sytuację początkową).
  \item Rozpoczyna się właściwa rozgrywka.
\end{enumerate}

\chapter{Rozgrywka}
\section{Lot podczas samouczka //To jest słabe, ale nie umiem lepiej, liczę na protipy}
\begin{enumerate}
  \item W~ momencie rozpoczęcia dowolnej z~ lekcji samouczka program wyświetla dodatkowe okno na~ komunikaty szkoleniowe oraz~ mini-zadania do wykonania przez użytkownika.
  \item Co~ pewien czas użytkownik dostaje komunikaty pozwalający zapoznać się z~ pewnym elementem sztuki prowadzenia samolotu.
  \item Po~ każdym komunikacie szkoleniowym użytkownik dostaje mini-zadanie do~ wykonania (np. będąc jeszcze na~ ziemi wychylić maksymalnie lotki i~ wrócić potem do~ początkowej pozycji).
  \item Po~ poprawnym wykonaniu mini-zadania program zapisuje stan gry i~ wyświetla kolejny komunikat.
  \item W~ przypadku niepowodzenia (trudność mini-zadań wzrasta wraz z~ postępami czynionymi w samouczku) program automatycznie wczytuje ostatni zapisany stan gry.
\end{enumerate}

\section{Lot podczas wykonywania misji //To też}
\begin{enumerate}
  \item W~ momencie rozpoczęcia misji program wyświetla dodatkowe okno z~ celami misji (głównymi i~ pobocznymi).
  \item Gdy~ użytkownik wykona jeden z~ celów misji, jest on~ zaznaczany na~ zielono.
  \item Gdy~ użytkownik podejmie akcję, która uniemożliwi wykonanie któregos z~ celów, to~ cel ten zostaje zaznaczony na~ czerwono.
  \item Po~ ukończeniu rozgrywki program pokazuje podsumowanie wyników misji i~, w przypadku pomyslnego zakończenia, nagradza użytkowika odpowiednimi trofea.
\end{enumerate}

\section{Sterowanie za pomocą klawiatury}
\begin{enumerate}
  \item Podczas lotu użytownik naciska dowolny klawisz uwzględniony w tabeli 2.1 z dokumentu \cite{WYM}.
  \item Program odczytuje wciśnięty klawisz i~ wykonuje odpowiednią reakcję.
\end{enumerate}

\section{Sterowanie za pomocą joysticka}
\begin{enumerate}
  \item Podczas lotu użytownik naciska dowolny ruch joysticka uwzględniony w tabeli 2.2 z dokumentu \cite{WYM}.
  \item Program odczytuje wciśnięty klawisz i~ wykonuje odpowiednią reakcję.
\end{enumerate}

\section{Sterowanie za pomocą myszy}
\begin{enumerate}
  \item Podczas lotu użytownik klika myszą na dowolny element w kokpicie samolotu.
  \item Program sprawdza, który element kokpitu został użyty i~ wykonuje stosowną akcję.
\end{enumerate}

\section{Kontakt z wieżą kontroli lotów}
\begin{enumerate}
  \item Użytkownik uaktywnia okno kontroli lotów poprzez kliknięcie na~ nie.
  \item Program wyświetla użytkownikowi listę możliwych komunikatów.
  \item Użytkownik wybiera komunikat, który chce nadać.
  \item Komunikat zostaje wysłany do~ wieży kontoli lotów.
  \item Po~ chwili użytkownik dostaje odpowiedź.
  \item Jeśli odpowiedź kończy wymianę komunikatów, następuje powrót do trybu lotu, w przeciwnym wypadku powtarzane są~ kroki~ 2-6.
\end{enumerate}

\section{Powiadomienia z wieży kontroli lotów //Krzysiu, sprawdź czy to ma w ogóle sens!}
\begin{enumerate}
  \item Wieża nadaje komunikat (np. o~ zmianie miejsca lądowania) do~ samolotu.
  \item Przy oknie kontroli lotów pojawia się ikona z~ liczbą nieodebranych komunikatów.
  \item Użytkownik klika na~ ikonę i~ odczytuje wiadomości.
\end{enumerate}

\section{Korzystanie z biblioteki}
\begin{enumerate}
  \item W~ dowolnym momencie użytkownik przechodzi do~ menu i~ naciska przycisk ,,Biblioteka''.
  \item Program łączy się z~ internetową bazą dokumentów i~ wyświetla prostą wyszukiwarkę.
  \item Użytkownik wyszukuje odpowiedni dokument w sposób określony w~ rozdziale~ 2.1.4 w~ \cite{WYM}.
  \item Program pobiera dokument i~ wyświetla go użytkownikowi.
\end{enumerate}

\begin{thebibliography}{9}
  \bibitem{WYM} R. Hirsz, K. Wieczorek, K. Wróbel: \textit{Symulator lotu: Specyfikacja wymagań}, Wrocław, SPIO IIUwr 2014.
\end{thebibliography}

\end{document}
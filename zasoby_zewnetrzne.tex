\documentclass{mwrep}
\usepackage{polski}
\usepackage[polish]{babel}
\usepackage[utf8]{inputenc}
\usepackage{tgbonum}
\usepackage{graphicx}
\usepackage{tabularx}
\usepackage{hyperref}

\makeatletter
\let\ps@closing\ps@plain
\makeatother

\renewcommand{\labelitemi}{$\bullet$}
\setlength{\parindent}{0cm}
\addto\captionspolish{\renewcommand{\tablename}{Tabela}}

\newcommand*{\titleGP}{\begingroup
\centering

{\large Studencka Pracownia Inżynierii Oprogramowania}\\Instytut Informatyki Uniwersytetu Wrocławskiego\par
\vspace*{16\baselineskip}

{\Large Rafał Hirsz, Karol Wieczorek, Krzysztof Wróbel\par}
\vspace*{\baselineskip}

\rule{\textwidth}{1.6pt}\vspace*{-\baselineskip}\vspace*{2pt}
\rule{\textwidth}{0.4pt}\\[\baselineskip]

{\Huge SYMULATOR LOTU}\\[0.2\baselineskip]

\rule{\textwidth}{0.4pt}\vspace*{-\baselineskip}\vspace{3.2pt}
\rule{\textwidth}{1.6pt}\\[\baselineskip]

\scshape
{\huge Zasoby zewnętrzne}\par
\vspace*{2\baselineskip}

\begin{figure}[h]
\centering
\includegraphics[width=5\baselineskip]{flightsim-team-logo.pdf}
\end{figure}
\vfill

{\large Wrocław 2014}\par

\pagebreak

\endgroup}

\begin{document}

\thispagestyle{empty}
\titleGP

\begin{center}
\begin{table}[h]
\begin{center}
\caption{Historia zmian dokumentu}\label{T:Zmiany}
\vspace{3ex}
\begin{tabularx}{1\textwidth}{|l|l|l|X|}
\hline
Data & Wersja & Autor & Opis zmian \\ \hline
2014-01-22 & 1.0 & Krzysztof Wróbel & Utworzenie dokumentu \\
\hline
\end{tabularx}
\end{center}
\end{table}
\end{center}

\pagebreak

\tableofcontents

\chapter{Zasoby statyczne}

\section{Mapy wysokości terenu}
Wysokość wyświetlanego terenu jest obliczana na podstawie dostępnych map wysokośći DTED (ang. \emph{Digital Terrain Elevation Data}. Dane te są opracowane podczas misji SRTM (ang. \emph{Shuttle Radar Topography Mission} promu kosmicznego \emph{Endevour}. 


\end{document}
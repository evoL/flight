\documentclass{mwrep}
\usepackage{polski}
\usepackage[polish]{babel}
\usepackage[utf8]{inputenc}
\usepackage{tgbonum}
\usepackage{graphicx}
\usepackage{tabularx}
\usepackage{hyperref}
\usepackage{rotating}

\makeatletter
\let\ps@closing\ps@plain
\makeatother

\renewcommand{\labelitemi}{$\bullet$}
\setlength{\parindent}{0cm}
\addto\captionspolish{\renewcommand{\tablename}{Tabela}}

\newcommand*{\titleGP}{\begingroup
\centering

{\large Studencka Pracownia Inżynierii Oprogramowania}\\Instytut Informatyki Uniwersytetu Wrocławskiego\par
\vspace*{16\baselineskip}

{\Large Rafał Hirsz, Karol Wieczorek, Krzysztof Wróbel\par}
\vspace*{\baselineskip}

\rule{\textwidth}{1.6pt}\vspace*{-\baselineskip}\vspace*{2pt}
\rule{\textwidth}{0.4pt}\\[\baselineskip]

{\Huge SYMULATOR LOTU}\\[0.2\baselineskip]

\rule{\textwidth}{0.4pt}\vspace*{-\baselineskip}\vspace{3.2pt}
\rule{\textwidth}{1.6pt}\\[\baselineskip]

\scshape
{\huge Założenia ogólne}\par
\vspace*{2\baselineskip}

\begin{figure}[h]
\centering
\includegraphics[width=5\baselineskip]{flightsim-team-logo.pdf}
\end{figure}
\vfill

{\large Wrocław 2014}\par

\pagebreak

\endgroup}

\newcommand{\fixedspaceword}[2][1]{%
  \begingroup
  \spaceskip=#1\fontdimen2\font minus \fontdimen4\font
  \xspaceskip=0pt\relax % just to be sure
  #2%
  \endgroup
}

\begin{document}

\thispagestyle{empty}
\titleGP

\begin{center}
\begin{table}[h]
\begin{center}
\caption{Historia zmian dokumentu}\label{T:Zmiany}
\vspace{3ex}
\begin{tabularx}{1\textwidth}{|l|l|l|X|}
\hline
Data & Wersja & Autor & Opis zmian \\ \hline
2013-11-05 & 1.0 & Zespół & Utworzenie dokumentu \\
2013-11-26 & 1.1 & Karol Wieczorek & \fixedspaceword{Uzupełnienie harmonogramu} \\
2014-01-04 & 1.2 & Krzysztof Wróbel & \fixedspaceword{Poprawki stylistyczne}, \fixedspaceword{uzupełnienie słownika pojęć} \\
2014-01-21 & 1.3 & Rafał Hirsz & \fixedspaceword{Poprawki stylistyczne}, \fixedspaceword{uzupełnienie słownika pojęć} \\
\hline
\end{tabularx}
\end{center}
\end{table}
\end{center}

\pagebreak

\tableofcontents

\chapter{Słownik pojęć\protect\footnotemark[1]{}}

\footnotetext[1]{Hasła opracowano na podstawie \emph{Wikipedii} (\url{http://pl.wikipedia.org/})}

\begin{description}
\item[COCOMO II] -- jedna z~metod szacowania pracochłonności systemu informatycznego bazująca na rozmiarze oprogramowania.
\item[DME] -- (ang. \emph{Distance Measuring Equipment}) radiowa pomoc nawigacyjna służąca do pomiaru fizycznej odległości pomiędzy samolotem a~stacją naziemną.
\item[ILS] -- (ang. \emph{Instrumental Landing System}) radiowy system nawigacyjny wspomagający lądowanie samolotu w~warunkach ograniczonej widoczności.
\item[Kąt natarcia] -- (ang. \emph{angle of attack}) jest to (umowny) kąt pomiędzy kierunkiem strugi napływającego powietrza, a~cięciwą skrzydła.
\item[Klapy] -- (ang. \emph{flaps}) ruchome powierzchnie sterowe, umiejscowione zazwyczaj w~tylnej części skrzydła statku powietrznego, pozwalająca w~razie potrzeby znacznie zwiększyć siłę nośną oraz opór skrzydła.
\item[Kod lotniska IATA] -- trzyliterowy kod alfanumeryczny, służący do oznaczania portów lotniczych na całym świecie. Kody te nadawane są przez Międzynarodowe Zrzeszenie Przewoźników Powietrznych (IATA).
\item[Kod lotniska ICAO] -- czteroliterowy kod, służący do adresowania portów lotniczych na całym świecie. Kody te nadawane są przez Organizację Międzynarodowego Lotnictwa Cywilnego (ICAO).
\item[Krawędź natarcia] -- (ang. \emph{leading edge}) przednia krawędź skrzydła.
\item[Krawędź spływu] -- (ang. \emph{trailing edge}) tylna część skrzydła, gdzie zbiegają się strugi powietrza znad i~spod skrzydła.
\item[Lot IFR] -- (ang. \emph{Instrumental Flight Rules}) lot wykonywany zgodnie z przepisami dla lotów według wskazań przyrządów.
\item[Lot VFR] -- (ang. \emph{Visual Flight Rules}) lot wykonywany zgodnie z przepisami dla lotów z widocznością.
\item[Lotki] -- (ang. \emph{ailerons}) powierzchnie sterowe statku powietrznego, służące do kontroli jego przechylenia.
\item[Metoda punktów funkcyjnych] -- najczęściej stosowana metoda szacowania wielkości systemów informatycznych uzależniająca ich złożoność od ilości realizowanych przez nie funkcji.
\item[NDB] -- (ang. \emph{Non-Directional Beacon}) naziemna radiolatarnia bezkierunkowa.
\item[Radiowysokościomierz] -- rodzaj lotniczego przyrządu pokładowego, \linebreak który wskazuje rzeczywistą wysokość lotu nad powierzchnią Ziemi.
\item[Skrzela] -- (ang. \emph{slots}) elementy skrzydła samolotu umieszczone wzdłuż krawędzi natarcia, mają za zadanie poprawienie własności aerodynamicznych statku powietrznego przy locie z~małymi prędkościami i~na dużych kątach natarcia poprzez zwiększenie siły nośnej lub opóźnienie oderwania się strug.
\item[Spojlery] -- (ang. \emph{spoilers}) element skrzydła, którego wychylenie skutkuje w~zmniejszeniu siły nośnej i~zwiększeniu tarcia skrzydła.
\item[System odniesienia WGS-84] -- zbiór parametrów (z~1984) określających wielkość i~kształt Ziemi oraz właściwości jej potencjału grawitacyjnego. System ten definiuje elipsoidę, która jest uogólnieniem kształtu geoidy, używaną do tworzenia map.
\item[Sztuczny horyzont] -- przyrząd pomagający w~określeniu orientacji przestrzennej samolotu względem horyzontu.
\item[TAWS] -- (ang. \emph{Terrain Awareness and Warning System}) urządzenie pokładowe ostrzegające pilotów o~zbliżającej się powierzchni gruntu.
\item[TCAS] -- (ang. \emph{Traffic Alert and Collision Avoidance System}) pokładowy system zapobiegający zderzeniom statków powietrznych.
\item[Teselacja] -- (ang. \emph{tesselation}) dzielenie wygenerowanych podczas tworzenia trójwymiarowego obrazu wielokątów na mniejsze, dzięki czemu wyświetlany obiekt może być dokładniej narysowany.
\item[Trymer] -- (ang. \emph{trim tab}) wychylna część steru na jego krawędzi spływu, ustawiana w~locie pod odpowiednim kątem, wychylana w~przeciwnym kierunku niż ster, zmniejsza siłę potrzebną pilotowi do utrzymania steru w~stałym wychyleniu.
\item[VOR] -- (ang. \emph{VHF Omni-directional Range}) rodzaj radiolatarni stosowanej w~lotnictwie.
\item[Wariometr] -- przyrząd służący do wskazywania prędkości pionowej \linebreak (wznoszenia lub opadania) statku.
\item[Wysokościomierz barometryczny] -- czuły, wyskalowany przyrząd \linebreak wskazujący wysokość dzięki pomiarowi ciśnienia powietrza.
\end{description}

\chapter{Specyfikacja symulatora lotu}

\section{Ogólna specyfikacja symulatora}

Symulator lotu jest grą symulującą ruch cywilnych maszyn latających, przeznaczoną dla komputerów osobistych klasy PC. Jego założeniem jest jak najwierniejsze modelowanie zjawisk fizycznych towarzyszących statkowi powietrznemu w~czasie lotu, jak i~na lotnisku. Dodatkowo kładzie się nacisk na dokładne odzwierciedlenie specyfiki ruchu lotnicznego i~pracy pilota.

\section{Komponenty symulatora}

\subsection{Symulator oddziaływań fizycznych}

Podstawowym celem symulatora jest dokładne wymodelowanie zjawisk fizycznych. Konieczna do tego jest implementacja w~grze mechanizmów dynamiki newtonowskiej. Symulowanie sił działających na bryłę sztywną powinno umożliwiać kontrolowanie jej toru ruchu, prędkości i~przyspieszenia. Nie zamierza się uwzględniać fizyki relatywistycznej.

\subsection{Modelowanie ruchu samolotu}

Statek powietrzny jest w~sensie fizycznym bryłą sztywną o~zmieniających się w~czasie ruchu parametrach kształtu i~masy. Działają na niego siły różnego pochodzenia. Można wśród nich wymienić:
\begin{description}
\item[Grawitację ziemską] -- w~zależności od aktualnego położenia obiektu na Ziemi i~wysokości nad poziomem morza przyspieszenie ziemskie może mieć minimalnie różne wartości. Spowodowane to jest odległością od środka Ziemi i~siły odśrodkowej wywołanej jej ruchem obrotowym.
\item[Siłę nośną] -- skrzydła samolotu poruszające się w~ściśliwym ośrodku takim, jakim jest powietrze, wytwarzają siłę przeciwną do siły grawitacji. Pozwala to na unoszenie się obiektu nad Ziemią. Wartość siły nośnej zależy od profilu skrzydła, jego powierzchni, prędkości w~ośrodku, kąta natarcia, ciśnienia atmosferycznego, temperatury, wychylenia klap i~skrzeli (slotów). Obroty samolotu wokół jego osi są powodowane przez różnice w~sile nośnej wytwarzanej przez poszczególne powierzchnie. Wychylanie lotek powoduje różnice siły nośnej na skrzydłach, co skutkuje przechylaniem się samolotu na boki. Usterzenie na statecznikach pionowych i~poziomych obraca samolot odpowiednio w~osi pionowej i~wzdłuż skrzydeł. Pozycja poszczególnych powierzchni sterujących może być utrzymana poprzez użycie trymeru.
\item[Opór powietrza] -- ruch obiektu w~ośrodku gazowym wymaga użycia siły do przesuwania napotkanych molekułów. Siła ta rośnie wraz ze wzrostem gęstości ośrodka i~prędkości przesuwania się w~nim.
\item[Ciąg] -- siła ciągu jest wynikiem działania silników. Planuje się implementacje trzech różnych rodzajów napędu: śmigłowy, odrzutowy i~rakietowy. Moc ciągu zależy od tempa spalania paliwa, gęstości powietrza i~od indywidualnych właściwości silnika. Przy locie wznoszącym lub opadającym ciąg ma wpływ na siłę nośną statku.
\item[Hamowanie] -- statek powietrzny może być wyhamowywany na cztery różne sposoby: przez użycie spojlerów, odwracaczy ciągu, hamulców w~kołach (tylko po przyziemieniu) oraz przez odpowiednie manewrowanie całym samolotem.
\item[Kontakt z~podłożem] -- podwozie samolotu działa podobnie do sprężyny. Przyziemienie powoduje przyłożenie do kadłuba siły skierowanej pionowo w~górę.
\end{description}

\vspace*{\baselineskip}
Ważną informacją potrzebną do symulacji ruchu samolotu jest jego rozkład masy. W~szczególności na zmianę zachowania statku w~powietrzu ma wpływ ilość oraz rozmieszczenie paliwa, pasażerów i~bagaży. Podczas lotu zmienia się także ilość paliwa w~zbiornikach.

\subsection{Warunki pogodowe}
W symulatorze oferowane będą trzy tryby pogody:

\begin{itemize}
\item warunki predefiniowane przez użytkownika,
\item warunki zgodne z~lokalnym klimatem, porą roku i~dnia,
\item warunki aktualne (pobierane z~zewnętrznych źrodeł).
\end{itemize}

W~każdym z~przypadków, wpływ na rzeczywiste odczyty temperatury, wilgotności i~ciśnienia ma wysokość, na której znajduje się samolot.


\subsection{Wykrywanie kolizji}
Ze względów etycznych po rozbiciu samolotu zostanie wygenerowany jedynie komunikat mówiący o~jego zniszczeniu. Nieprzewidywane są żadne animacje katastrofy i~wraku. Kolizje z~terenem poza miejscem lądowania będą wykrywane mniej dokładnie oraz będą od razu przerywały symulację lotu. W~przypadku lotnisk informacja o~kolizji będzie wyświetlana w~wyniku przeciążenia podwozia lub dotknięcia ziemi innym elementem maszyny. Zarówno zbliżenia do terenu lotniska, jak i~do innych obiektów będą symulowane z~najwyższą szczegółowością.

\subsection{Dane techniczne samolotów}
Program ma umożliwiać korzystanie z~różnych modeli samolotów. Niekoniecznie muszą one być dostarczone przez wytwórcę symulatora. Dane konfiguracyjne powinny spełniać specyfikację ze strony \url{http://msdn.microsoft.com/en-us/library/cc526949.aspx}.

\subsection{Topografia Ziemi}
Przewiduje się symulację lotu w~realnym świecie. Ukształtowanie terenu, jeziora, morza i~położenie lotnisk będą czytane z~map. Kształt Ziemi powinien być elipsoidą obrotową spłaszczoną na biegunach, por. system odniesienia WGS-84.

\subsection{Przyrządy nawigacyjne}
Symulator powinien modelować zachowanie się niektórych przyrządów nawigacyjnych, m.in. map, kompasu, wysokościomierza barometrycznego, radiowysokościomierza, prędkościomierza, wariometru, sztucznego horyzontu oraz systemów ILS, VOR, DME, NDB, TCAS, TAWS i~GPS.

\subsection{Generowanie grafiki}
W programie umożliwiony będzie podgląd świata z~kabiny pilota, jak i~z zewnątrz samolotu. Sceneria będzie przedstawiała podstawową rzeźbę terenu, urządzenia na lotnisku oraz inne pojazdy i~statki powietrzne. Dodatkowo podczas widoku kokpitu przedstawione bedą wszystkie przyrządy pokładowe.

\subsection{Interakcja z~użytkownikiem}
Użytkownik będzie mógł sterować statkiem, używając myszy, joysticka i~klawiatury. Mysz będzie pomagała w~rozglądaniu się, joystick w~ręcznym sterowaniu samolotem, a~klawiatura w~szczegółowych ustawieniach lotu.

\subsection{Kontrola lotów}
Osobną pomocą nawigacyjną jest automatyczna kontrola lotów. Będzie ona dokonywana w~trybie tekstowym w~angielskim języku lotniczym.

\subsection{Inne pojazdy}
Implementacji wymaga ruch służb lotniskowych, np. cystern, samochodów, holowników, itd.

\subsection{Planowanie lotu}
Ważnym elementem przed startem maszyny jest zaplanowanie lotu. Symulator będzie umożliwiał graficzne zaplanowanie trasy przelotu i~ilości paliwa.

\subsection{Interfejs użytkownika}
Interfejs użytkownika powinien umożliwiać skonfigurowanie w trybie okienkowym parametrów symulacji. W~szczególności chodzi tu o: wybór samolotu, dnia, godziny, miejsca startu, pogody, trasy lotu, ilości paliwa i~liczby pasażerów.

\subsection{Gra online}
Użytkownicy będą mogli współpracować ze sobą i~widzieć się nawzajem, grając poprzez Sieć.

\section{Scenariusze użycia}
Przewiduje się pięć różnych trybów gry.
\subsection{Lot dowolny}
W tym trybie jedynym celem użytkownika jest pilotowanie samolotu, wykonywanie akrobacji, testowanie osiągów, trening pilotażu w~trudnych warunkach pogodowych, itp. Nie buduje się planu lotu, ani nie działa kontrola lotów. W~przestrzeni nie ma także innych statków powietrznych.
\subsection{Przelot według planu lotu}
Scenariusz ten najbardziej odzwierciedla realną pracę pilota. Gra zaczyna się przy hangarze na płycie lotniska. Użytkownik powinien utworzyć plan lotu, zatankować, zabrać ładunek lub pasażerów, uzyskać pozwolenie na lot i~start, stosować się do poleceń kontroli lotów i~prawa lotniczego. Może napotkać inne, kontrolowane przez komputer, samoloty.
\subsection{Tryb wielu graczy}
Symulacja podobna jest do opisanej w p. 2.3.2 z~tą różnicą, że uczestniczyć w~niej może kilku rzeczywistych użytkowników.
\subsection{Samouczek}
Tryb treningowy pozwalający na praktykę sztuki pilotażu i~poznawanie pomocy nawigacyjnych.
\subsection{Misje}
Specjalne, z góry określone zadania do wykonania przez gracza. Za ich ukończenie przyznawane będą trofea.

\section{Opis użytkownika}
Z definicji gra jest dokładnym symulatorem i~jej celem nie jest zabawianie, tylko dokładne modelowanie rzeczywistości. Od użytkownika oczekuje się precyzji, cierpliwości, opanowania, szybkiej reakcji, zdolności manualnych oraz chęci do nauki i znajomości języka angielskiego.

\chapter{Harmonogram}
Harmonogram ,,Symulatora lotu'' został sporządzony przy użyciu powszechnie stosowanych metod szacowania rozmiaru i~pracochłonności oprogramowania oraz na podstawie wiedzy członków zespołu. Do wyznaczenia wielkości projektu użyto metody punktów funkcyjnych i~na jej podstawie stwierdzono, że kod programu będzie miał ok. 50-60 tysięcy wierszy. Następnie, za pomocą metody COCOMO~II obliczono przybliżoną pracochłonność całego projektu (wyniosła ok. 180 osobomiesięcy) oraz jego poszczególnych części składowych. Do tego wyniku dodano 20 osobomiesięcy buforu związanego z urlopami pracowników i~innymi niemożliwymi do przewidzenia zdarzeniami. Przy założeniu, że zespół programistów będzie liczył 10 osób, wykonanie oprogramowania powinno zająć niecałe dwa lata.
Poniższy harmonogram (tab. \ref{harmonogram}) przedstawia planowany przebieg prac nad projektem i~ze względu na~wczesną fazę prac jest obłożony dużą niedokładnością. Jego dokładniejsza wersja powstanie w~fazie analizy.

\begin{sidewaystable}
    \caption{Koszt poszczególnych etapów kontrukcji oprogramowania}
    \begin{center}
        \begin{tabular}{|p{7cm}|r|r|r|r|r|} \hline
            \multicolumn{1}{|m{7cm}|}{\centering Funkcja}                                                                        &
            \multicolumn{1}{m{2,5cm}|}{\centering Koszt w~punktach funkcyjnych}                                                  &
            \multicolumn{1}{m{4cm}|}{\centering Koszt w~punktach funkcyjnych po~uwzględnieniu wsp. wpływu}                       &
            \multicolumn{1}{m{2,5cm}|}{\centering Liczba wierszy kodu}                                                           &
            \multicolumn{1}{m{2,5cm}|}{\centering Osobomiesiące}                                                                 &
            \multicolumn{1}{m{4cm}|}{\centering Osobomiesiące z~uwzględnieniem buforu na~nieprzewidziane zdarzenia}              \\ \hline

            Implementacja symulacji dynamiki newtonowskiej                      & 118     & 112     & 5935     & 20,49   & 22,78 \\
            Implementacja warunków pogodowych                                   & 124     & 118     & 6257     & 21,70   & 24,13 \\
            Wyświetlanie warunków pogodowych                                    & 49      & 47      & 2484     & 7,93    & 8,81  \\
            Implementacja i wyświetlanie zmiennych warunków pogodowych          & 58      & 55      & 2898     & 9,38    & 10,43 \\
            Implementacja podstawowego mechanizmu lotu                          & 57      & 54      & 2852     & 9,22    & 10,24 \\
            Implementacja podstawowego mechanizmu graficznego                   & 62      & 59      & 3128     & 10,19   & 11,33 \\
            Wykrywanie kolizji                                                  & 51      & 49      & 2576     & 8,25    & 9,17  \\
            Wczytanie i wyświetlanie map                                        & 98      & 93      & 4923     & 16,71   & 18,58 \\
            Implementacja infrastruktury pomocniczej                            & 92      & 88      & 4647     & 15,69   & 17,44 \\
            Wyświetlanie infrastruktury pomocniczej                             & 45      & 43      & 2254     & 7,13    & 7,93  \\
            Implemantacja interfejsu użytkownika                                & 59      & 56      & 2990     & 9,70    & 10,79 \\
            \fixedspaceword{Implementacja obługi alternatywnych kontrolerów}    & 45      & 43      & 2254     & 7,13    & 7,93  \\
            Wczytanie modeli samolotów                                          & 101     & 95      & 5061     & 17,22   & 19,15 \\
            Implementacja lądowania                                             & 107     & 102     & 5383     & 18,42   & 20,48 \\
            \hline
            \hfill suma                                                         & 1065    & 1012    & 53642    & 179,17  & 199,17\\ \hline
        \end{tabular}
    \end{center}
\end{sidewaystable}

\begin{table}
    \caption{Współczynniki wpływu dla metody punktów funkcyjnych}
    \begin{center}
        \begin{tabular}{|p{10cm}|r|} \hline
        \multicolumn{1}{|m{10cm}|}{\centering Współczynniki wpływu }                                           &
        \multicolumn{1}{m{2cm}|}{\centering Punkty}                                                            \\ \hline

        Czy wymagane jest przesyłanie danych?                                                           & 4    \\
        Czy występują funkcje przetwarzania rozproszonego?                                              & 0    \\
        Czy wydajność ma kluczowe znaczenie?                                                            & 4    \\
        Czy system ma działać w mocno obciążonym środowisku operacyjnym?                                & 3    \\
        Czy system wymaga wprowadzania danych online?                                                   & 0    \\
        Czy wewnętrzne przetwarzanie danych jest złożone?                                               & 5    \\
        Czy kod ma być przystosowany do ponownego użycia?                                               & 2    \\
        Czy wejścia, wyjścia, pliki i zapytania są złożone?                                             & 3    \\
        Czy wprowadzenie danych on-line wymaga transakcji obejmujących wiele ekranów lub operacji?      & 3    \\
        Czy pliki główne są aktualizowane on-line?                                                      & 2    \\
        Czy system ma mieć automatyczne konwersje i instalacje?                                         & 4    \\
        Czy system wymaga mechanizmu kopii zapasowych i odtwarzania?                                    & 3    \\
        Czy system jest projektowany dla wielu instalacji w różnych przedsiębiorstwach?                 & 4    \\
        Czy aplikacja jest projektowana, aby wspomagać zmiany i być łatwą w użyciu przez użytkownika?   & 2    \\ \hline
        \hfill Suma                                                                                     & 39   \\ \hline
        \hfill Współczynnik wpływu                                                                      & 0,95 \\ \hline

        \end{tabular}
    \end{center}
\end{table}

\begin{table}
    \caption{Współczynniki metody COCOMO II}
    \begin{center}
        \begin{tabular}{|l|l|} \hline
        \multicolumn{1}{|m{10cm}|}{\centering Współczynniki COCOMO II } &
        \multicolumn{1}{m{2cm}|}{\centering Punkty}                    \\ \hline
        Typowość                        & 4,96 \\
        Elastyczność                    & 3,04 \\
        Zarządzanie ryzykiem            & 4,24 \\
        Spójność zespołu                & 1,1  \\
        Dojrzałość procesu wytwarzania  & 4,68 \\ \hline
        \end{tabular}
    \end{center}
\end{table}

\begin{table}
	\caption{Harmonogram projektu\label{harmonogram}}
  \centerline{\includegraphics*[scale=0.8]{harmonogram-tabela.pdf}}
\end{table}

\chapter{Kosztorys}

\section{Zespół}

Planujemy oddać część pracy w~ręce zewnętrznych podwykonawców. Dlatego dokonujemy podziału zespołu na zespół stacjonarny i~niestacjonarny.

\subsection{Zespół stacjonarny}

\begin{itemize}
\item 10 programistów C++,
\item 2 kierowników zespołów,
\item 4 testerów oprogramowania.
\end{itemize}

\subsection{Zespół niestacjonarny}

\begin{itemize}
\item 2 grafików 2D,
\item 2 grafików 3D.
\end{itemize}

\section{Miesięczne wynagrodzenia brutto pracowników}

Poniższa lista została sporządzona na podstawie danych pochodzących z raportu \emph{Wynagrodzenia na stanowiskach IT w~2012 roku} wydanego przez kancelarię Sedlak \& Sedlak.\footnote{\url{http://wynagrodzenia.pl}} \\

\begin{tabular}{lr}
programista C++ & 6500 zł \\
kierownik zespołu & 8000 zł \\
grafik 3D & 7400 zł \\
grafik 2D & 3700 zł \\
tester oprogramowania & 4200 zł
\end{tabular}

\section{Koszt utrzymania pracowników}

\begin{tabular}{llr}
programiści C++ & 20 miesięcy & 1 300 000 zł \\
kierownik zespołu & 20 miesięcy & 160 000 zł \\
graficy 3D & 13 miesięcy & 192 400 zł \\
graficy 2D & 13 miesięcy & 96 200 zł \\
testerzy & 20 miesięcy & 336 000 zł \\
\hline
\textbf{suma} && \textbf{2 084 600 zł}
\end{tabular}

\section{Koszt całkowity}

\begin{tabular}{lr}
wynagrodzenia & 2 084 600 zł \\
bufor na nieprzewidziane wydatki & 40 000 zł \\
\hline
\textbf{suma} & \textbf{2 124 600 zł}
\end{tabular}

\end{document}
\documentclass{mwrep}
\usepackage{polski}
\usepackage[polish]{babel}
\usepackage[utf8]{inputenc}
\usepackage{tgbonum}
\usepackage{graphicx}
\usepackage{tabularx}
\usepackage{hyperref}

\renewcommand{\labelitemi}{$\bullet$}
\setlength{\parindent}{0cm}
\addto\captionspolish{\renewcommand{\tablename}{Tabela}}

\newcommand*{\titleGP}{\begingroup
\centering

{\large Studencka Pracownia Inżynierii Oprogramowania}\\Instytut Informatyki Uniwersytetu Wrocławskiego\par
\vspace*{16\baselineskip}

{\Large Rafał Hirsz, Karol Wieczorek, Krzysztof Wróbel\par}
\vspace*{\baselineskip}

\rule{\textwidth}{1.6pt}\vspace*{-\baselineskip}\vspace*{2pt}
\rule{\textwidth}{0.4pt}\\[\baselineskip]

{\Huge SYMULATOR LOTU}\\[0.2\baselineskip]

\rule{\textwidth}{0.4pt}\vspace*{-\baselineskip}\vspace{3.2pt}
\rule{\textwidth}{1.6pt}\\[\baselineskip]

\scshape
{\huge Specyfikacja wymagań}\par
\vspace*{2\baselineskip}

\begin{figure}[h]
\centering
\includegraphics[width=5\baselineskip]{flightsim-team-logo.pdf}
\end{figure}
\vfill

{\large Wrocław 2013}\par

\pagebreak

\endgroup}

\begin{document}

\thispagestyle{empty}
\titleGP

\begin{center}
\begin{table}[h]
\begin{center}
\begin{tabularx}{1\textwidth}{|l|l|l|X|}
\hline
Data & Wersja & Autor & Opis zmian \\ \hline
2013-12-11 & 1.0 & Krzysztof Wróbel & Utworzenie dokumentu \\
\hline
\end{tabularx}
\end{center}
\vspace{3ex}
\caption{Historia zmian dokumentu}\label{T:Zmiany}
\end{table}
\end{center}

\pagebreak

\tableofcontents

\chapter{Charakterystyka użytkowników}

Symulator jest grą komputerową przeznaczoną zarówno dla użytkowników z zerową wiedzą o lotnictwie, jak i dla profesjonalnych pilotów. W zależności od stopnia wtajemniczenia gracza w zagadnienia awioniki oferuje różne funkcje dostosowane do jego umiejętności.

\section{Nowicjusz}

Scenariusz gry dla osoby nieposiadającej wiedzy z zakresu lotnictwa przewiduje lekcje podstawowych technik pilotażu lekkich samolotów. Lekcje dotyczą startu i lądowania, lotu poziomego, zakrętów, podchodzenia do lądowania i kołowania. Dostępne są także podstawowe misje zręczonościowe oraz swobodny lot.

\section{Entuzjasta lotnictwa}

Zakłada się, że entuzjaści lotnictwa chcą maksymalnie zgłębić swoją wiedzę o pilotażu. Grupie tej oferowany jest pełny cykl artykułów, samouczków i lekcji z następujących dziedzin: ogólna wiedza o samolocie, prawo lotnicze, zasady ruchu lotniczego, osiągi i planowanie lotu, meteorologia, nawigacja, procedury operacyjne, zasady lotu, łączność oraz ogólne bezpieczeństwo lotu. Przygotowane są misje o narastającym poziomie trudności. Przewiduje się, że po ukończeniu kursu entuzjaści będą w stanie wykonać zadania przeznaczone dla pilotów w realnym świecie.

\section{Pilot w realnym świecie}

Program z założenia ma jak najlepiej symulować rzeczywisty lot. Przy jego pomocy użytkownicy z licencją pilota mogą cieszyć się realną symulacją przy nieporównywalnie niższych kosztach od podróży samolotem lub wynajęcia prawdziwego symulatora. Gracze mogą sprawdzić swoje umiejętności przy każdych warunkach pogodowych w każdym miejscu na Ziemi. Dodatkowo zaaranżowane mogą zostać nieprzewidziane usterki oraz problemy z pasażerami na pokładzie. Licencjonowany pilot będzie używał wszystkich elementów gry. W szczególności jest tutaj mowa o dokładnym zaplanowaniu trasy lotu i załadowania samolotu, manewrach na lotnisku, kontroli lotów, nawigowaniu, przestrzeganiu procedur lotniczych, używaniu map, schematów oraz list kontrolnych. Dla tej grupy graczy przygotowane są skomplikowane misje przedstawiające sytuacje z jakimi spotykają się prawdziwi piloci.

\chapter{Wymagania funkcjonalne}

\section{Konfiguracja gry}

\subsection{Konfiguracja swobodnego lotu}

Przed rozpoczęciem symulacji swobodnego lotu należy określić:
\begin{itemize}
\item używany samolot,
\item miejsce startu,
\item czas startu,
\item warunki pogodowe
\end{itemize}
oraz opcjonalnie:
\begin{itemize}
\item ładunek samolotu,
\item plan lotu,
\item ewentualne usterki.
\end{itemize}

\subsubsection{Wybór samolotu}
Gracz wybiera samolot na podstawie jego wytwórcy (np. Boeing), rodziny (np. 737) i modelu (np. 737-800). Może wybrać także jego malowanie (jeżeli jest dostępne więcej niż jedno). Podczas wyboru maszyny gracz powinien być poinformowany o jej podstawowych osiągach. Wybór przedstawiany jest graficznie.

\subsubsection{Wybór miejsca startu}
Użytkownik określa lotnisko, z którego ma się rozpocząć symulacja. Dostępne są następujące kryteria wyszukiwania:
\begin{itemize}
\item państwo (np. Grecja),
\item miasto (np. Londyn),
\item nazwa lotniska (np. Okęcie),
\item kod lotniska IATA (np. KRK),
\item kod lotniska ICAO (np. VHHH).
\end{itemize}
Dodatkowo na obszarze danego lotniska należy dospecyfikować miejsce startu symulacji, np.:
\begin{itemize}
\item pas startowy 18L,
\item pas startowy 36R,
\item bramka 12,
\item stanowisko odladzania,
\item hangar C.
\end{itemize}
Podane miejsca zależą od konkretnego portu lotniczego.

\subsubsection{Wybór czasu startu}
Symulacja wymaga określenia czasu jej startu (tj. daty i godziny).

\subsubsection{Wybór warunków pogodowych}
Użytkownik ma do wyboru następujące scenariusze pogodowe:
\begin{itemize}
\item predefiniowany -- wybiera się konkretny zestaw zjawisk atmosferycznych (np. mroźny, bezchmurny dzień lub ulewny deszcz),
\item odpowiedni dla miejsca i pory symulacji -- pogoda jest określana na podstawie klimatu miejsca startu,
\item aktualizowany -- pogoda jest aktualizowana na bieżące na podstawie zewnętrznych źródeł.
\end{itemize}

\subsubsection{Określenie ładunku samolotu}
Określenie ładunku samolotu składa się na podanie:
\begin{itemize}
\item liczby i rozmieszczenia pasażerów,
\item liczby i rozmieszczenia bagaży,
\item ilości i rozmieszczenia ładunku cargo,
\item ilości i rozmieszczenia paliwa.
\end{itemize}

\subsubsection{Stworzenie planu lotu}
W celu stworzenia planu lotu gracz musi podać następujące informacje:
\begin{itemize}
\item lotnisko docelowe -- wybierane w podobny sposób, co lotnisku startu,
\item tryb lotu:
\begin{itemize}
\item lot IFR,
\item lot VFR,
\end{itemize}
\item rodzaj nawigacji:
\begin{itemize}
\item na podstawie VOR i NDB,
\item korytarzami powietrznymi na niskiej wysokości,
\item korytarzami powietrznymi na wysokiej wysokości,
\item trasa bezpośrednia.
\end{itemize}
\end{itemize}
Wygenerowany plan lotu może zostać wydrukowany.

\subsubsection{Określenie usterek samolotu}
Gracz może wymusić, by podczas symulacji pojawiły się konkretne awarie poniższych typów:
\begin{itemize}
\item awarie przyrządów (np. wysokościomierza, sztucznego horyzontu),
\item awarie systemów (np. elektrycznego, próżniowego),
\item awarie radia,
\item awarie silników.
\end{itemize}
Dodatkowo istnieje możliwość spowodowania losowej usterki.

\subsection{Wybór misji}

Użytkownik ma możliwość wyboru konkretnego zadania do wykonania podczas symulacji, tzw. misji. Nazwy misji są posortowane według stopnia trudności. W podglądzie jest widoczny krótki opis i szacowany czas ukończenia. Po wybraniu misji graczowi przedstawiane są szczegóły, mapy i diagramy potrzebne do jej przejścia. Podczas wyboru misji nie ma możliwości konfiguracji swobodnego lotu.

\subsection{Szkoła}
W szkole dostępne są lekcje pilotażu w formie pisemnej (strony HTML) i symulacji (podobne do misji).

\subsection{Biblioteka}
Biblioteka jest bazą dokumentów potrzebnych pilotom do prowadzenia samolotów. W bibliotece znajdują się:
\begin{itemize}
\item instrukcje obsługi samolotów,
\item listy kontrolne,
\item schematy lotnisk,
\item schematy podejścia,
\item mapy stref przylotniskowych,
\item mapy przestrzeni powietrznych.
\end{itemize}

Dokumenty dotyczące samolotów powinny być wyszukiwane według modelu samolotu, dotyczące lotnisk -- według nazw miejscowości, lotniska lub kodu lotniska, natomiast mapy przestrzeni powietrznych -- według kraju.

\section{Symulacja lotu}

\subsection{Sterowanie}

W tabeli \ref{T:Sterowanie_klawiatura} zestawiono klawisze i skutki ich naciśnięcia. W tabeli \ref{T:Sterowanie_joystickiem} przedstawiono zdarzenia towarzyszące danym ruchom joysticka.\\
Symulator umożliwia dodatkowo sterowanie myszą. Klikanie w konkretne elementy kokpitu daje takie same efekty, jak użycie ich w prawdziwym samolocie.

\begin{center}
\begin{table}[ht]
\begin{center}
\begin{tabularx}{1\textwidth}{|l|X|}
\hline
Klawisz(e) & Skutek \\ \hline
\texttt{Esc} & Zakończenie symulacji \\
\texttt{P} & Pauza \\
\texttt{F1} & Ciąg zerowy \\
\texttt{F2} & Zmniejszenie ciągu \\
\texttt{F3} & Zwiększenie ciągu \\
\texttt{F4} & Ciąg maksymalny \\
\texttt{F5} & Schowanie klap i skrzeli \\
\texttt{F6} & Stopniowe chowanie klap i skrzeli \\
\texttt{F7} & Stopniowe wysuwanie klap i skrzeli \\
\texttt{F8} & Całkowite wysunięcie klap i skrzeli \\
\texttt{F9} & Widok zza samolotu \\
\texttt{F10} & Widok w dół \\
\texttt{F11} & Widok przodu samolotu \\
\texttt{F12} & Widok z kokpitu \\
\texttt{`} & Otwarcie/zamknięcie okna kontroli lotów \\
\texttt{1} & Otwarcie szczegółowego widoku sztucznego horyzontu \\
\texttt{2} & Otwarcie szczegółowego widoku przepustnic \\
\texttt{3} & Otwarcie szczegółowego widoku GPS \\
\texttt{4} & Otwarcie szczegółowego widoku radia \\
\texttt{5} & Otwarcie szczegółowego widoku panelu górnego \\
\texttt{/} & Wysunięcie/schowanie spojlerów \\
\texttt{.} & Zaciśnięcie/zwolnienie hamulców \\
\texttt{Z} & Włączenie/wyłączenie autopilota \\
\texttt{L} & Włączenie/wyłączenie świateł lądowania \\
\texttt{G} & Wysunięcie/schowanie podwozia \\
\texttt{E} & Uruchomienie silników \\
Strzałki & Poruszanie kamerą \\
\hline
\end{tabularx}
\end{center}
\vspace{3ex}
\caption{Sterowanie klawiaturą}\label{T:Sterowanie_klawiatura}
\end{table}
\end{center}

\begin{center}
\begin{table}[ht]
\begin{center}
\begin{tabularx}{1\textwidth}{|l|X|}
\hline
Zdarzenie & Skutek \\ \hline
Wychylenie do przodu & Opuszczenie krawędzi spływu statecznika poziomego \\
Wychylenie do tyłu & Opuszczenie krawędzie natarcia statecznika poziomego \\
Wychylenie w prawo & Wychylenie lotek na prawym i lewym skrzydle odpowiednio do dołu i do góry \\
Wychylenie w prawo & Wychylenie lotek na prawym i lewym skrzydle odpowiednio do góry i do dołu \\
Obrót w prawo & Wychylenie lotki na stateczniku pionowym w prawo \\
Obrót w lewo & Wychylenie lotki na stateczniku pionowym w lewo \\
\hline
\end{tabularx}
\end{center}
\vspace{3ex}
\caption{Sterowanie joystickiem}\label{T:Sterowanie_joystickiem}
\end{table}
\end{center}

\subsection{Kontrola lotów}

Kontrola lotów prowadzona jest w osobnym oknie dialogowym w trybie tekstowym. Językiem komunikatów jest język angielski. Graczowi są proponowane wypowiedzi, które po wyborze są przekazywane kontrolerowi. Kontrola lotów jest zaimplementowana zgodnie z tekstem \emph{Ustawy z dnia 3 lipca 2002 r. -- Prawo lotnicze} z późniejszymi poprawkami.

\chapter{Wymagania niefunkcjonalne}

\section{Realizm}
\subsection{Realizm fizyczny}
Kładzie się nacisk na dokładność realizacji silnika fizycznego. Obliczony tor ruchu bryły sztywnej, na którą działają siły zewnętrzne powinien w możliwie najmniejszym stopniu różnić się od teoretycznego modelu newtonowskiego.\\
Działanie poszczególnych części samolotu powinno być odwzorowane według najprecyzyjniejszych znanych modeli.
\subsection{Realizm graficzny}
Należy dokładnie wyrenderować modele samolotów, lotnisk, terenu, naziemnych pomocy nawigacyjnych, budynków, jak i urządzeń wewnątrz kokpitu.

\section{Niezawodność}
Zakłada się, że gracze będą chcieli odtwarzać najdłuższe możliwe przy aktualnej technologii loty, dlatego też program powinien być w stanie pracować przez okres do 48 godzin.

\section{Interfejsy programu}
Gracz, aby korzystać ze wszystkich funkcji symulatora, powinien posiadać sprawne łącze internetowe.

\section{Interfejs sprzętowy}
Minimalne wymagania:
\begin{itemize}
\item procesor: 1.33 GHz,
\item RAM: 1 GB,
\item karta graficzna: 128 MB,
\item miejsce na dysku twardym: 4 GB
\end{itemize}
\vspace{1ex}
Zalecane wymagania:
\begin{itemize}
\item procesor: dwurdzeniowy 2.0 GHz,
\item RAM: 4 GB,
\item karta graficzna: 256 MB,
\item miejsce na dysku twardym: 4 GB
\end{itemize}

\end{document}
\documentclass{mwrep}
\usepackage{polski}
\usepackage[polish]{babel}
\usepackage[utf8]{inputenc}
\usepackage{tgbonum}
\usepackage{graphicx}
\usepackage{tabularx}
\usepackage{hyperref}

\makeatletter
\let\ps@closing\ps@plain
\makeatother

\renewcommand{\labelitemi}{$\bullet$}
\setlength{\parindent}{0cm}
\addto\captionspolish{\renewcommand{\tablename}{Tabela}}
\setcounter{secnumdepth}{0}

\newcommand{\bitem}[1]{\item R. Hirsz, K. Wieczorek, K. Wróbel: \textit{Symulator lotu: #1}, Wrocław, SPIO IIUWr 2014.}

\newcommand*{\titleGP}{\begingroup
\centering

{\large Studencka Pracownia Inżynierii Oprogramowania}\\Instytut Informatyki Uniwersytetu Wrocławskiego\par
\vspace*{16\baselineskip}

{\Large Rafał Hirsz, Karol Wieczorek, Krzysztof Wróbel\par}
\vspace*{\baselineskip}

\rule{\textwidth}{1.6pt}\vspace*{-\baselineskip}\vspace*{2pt}
\rule{\textwidth}{0.4pt}\\[\baselineskip]

{\Huge SYMULATOR LOTU}\\[0.2\baselineskip]

\rule{\textwidth}{0.4pt}\vspace*{-\baselineskip}\vspace{3.2pt}
\rule{\textwidth}{1.6pt}\\[\baselineskip]

\scshape
{\huge Mapa dokumentów}\par
\vspace*{2\baselineskip}

\begin{figure}[h]
\centering
\includegraphics[width=5\baselineskip]{flightsim-team-logo.pdf}
\end{figure}
\vfill

{\large Wrocław 2014}\par

\pagebreak

\endgroup}

\begin{document}

\thispagestyle{empty}
\titleGP

\begin{center}
\begin{table}[h]
\begin{center}
\caption{Historia zmian dokumentu}\label{T:Zmiany}
\vspace{3ex}
\begin{tabularx}{1\textwidth}{|l|l|l|X|}
\hline
Data & Wersja & Autor & Opis zmian \\ \hline
2013-12-21 & 1.0 & Rafał Hirsz & Utworzenie dokumentu \\
\hline
\end{tabularx}
\end{center}
\end{table}
\end{center}

\section{Mapa dokumentów}

\begin{enumerate}
    \bitem{Założenia ogólne}
    \begin{enumerate}
        \item \textit{Słownik pojęć}
        \item \textit{Specyfikacja symulatora lotu}
        \item \textit{Harmonogram}
        \item \textit{Kosztorys}
    \end{enumerate}
    \bitem{Specyfikacja wymagań}
    \bitem{Architektura oprogramowania}
    \bitem{Przypadki użycia}
    \bitem{Plan testowania}
    \bitem{Zasoby zewnętrzne}
    \bitem{Mapa dokumentów}
\end{enumerate}

\end{document}